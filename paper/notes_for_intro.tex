General notes:

Outline - 
1 - Galaxy mergers are an importance feature of LCDM theory and are critical for the formation and evolution of galaxies across time. 
    - An important feature of galaxy mergers is the merger timescale...

2 - Merger timescales are important to calibrate 
    - They are crucial for understanding the observations of galaxy morphologies and understanding heirarchical evolution, (think galaxy start bursts, formation of gas bridges maybe, fueling agn, BH merger rates, etc)
    - Also important for understanding the few well modelled orbits that we do have, particularly in the LG (might come way later) 

3 - Merger timescales are also used to get merger rates, both observationally and theoretically, of close pairs.
    - How are close pairs defined? 
    - How do merger timescales enter merger rate calculations?
    - What is important about the merger rates that are used? 
    - Maybe need to make distinction here that there are also observability timescales, but they are going to be basically useless for TNG100 studies (RG2019 mocks i think?), but also are low mass pairs have a lot of gas and look super weird. 
    - RG 2019 looked at the simulated morphology of illustristng galaxies and found that for M*<1e9.5, the morphologies are "less reliable" -> motivates using pair fractions for merger rate studies 


4 - Comparing the merger rates of high mass and low mass pairs is important why?
    - Is important for our understanding of the build up of higher mass things in LCDM, and thus a test of LCDM
    - Different merger mechanics would be really interesting, i.e. if the DM halo profile (cuspy/core etc) impacted the merger processes

5 - Merger timescales have been studied a lot (mostly in obs and theory of high mass pairs)
    - Lotz 2011
    - RG2015
    - Snyder (maybe?) 
    - Jiang 2008 & BK 2008 for all things about the "Chandra" timescale for merging galaxies which scales with rc/Vc and is thus prop to the age of the universe (eq. 4\&5). Also quantified the circularity of orbits as a function of mass ratio <- might be good to mention
    - Jiang 2014 

    
6 - Problem is, we don't have a lot of studies on low mass pair merger rates? 
    - This is not ideal because it means that we typically extend merger rates calibrated for high mass down to low mass, but this means that we aren't able to test LCDM and draw conclusions about the impact

7 - In addition, our previous work showed that even the evolution of the fractions of high and low mass pairs can't be robustly studied and compared if even the selection criteria for the pairs are off. 
    - This means that the selection of pairs for pair fractions, merger fractions, and for merger timescales, are ALL going to be impacted by the selection criteria that's used. 
    - This is also a problem in observations already, since different samples will have different completeness down to different projected separations, and with different mass ratios, etc. 
    - O'Leary 2021 merging systems can be missed if you don't use wide enough sep criteria, thus leading to under predict merger rate

8 - Fortunately, our previous work showed that using scaled criteria for pair selection as a function of mass and redshift allows you to fairly compare samples. 
    - This motivated us to study the merger fraction and merger timescales of our galaxy pair sample from the first study, to see how merger timescales might ALSO depend on the selection criteria that are used to pick the pair sample. 


9 - Here, we aim to build a framework to understand the pair selection criteria that are used to select samples, and the merger timescales that result from various selections.  we look at the orbits of the pairs from our previous work, and determine the merger timescale for low and high mass pairs as a function of pair separation and redshift. In sec. 1... etc. 






Conclusions - 






Paper:
Author:
What's the connection:
Other notes:

PRIORITY: 
    Author: O'Leary 2021
    Paper: Emerge: Constraining merging probabilities and timescales of close galaxy pairs 
    What's the connection:
    Other notes: 
    Empirical models (EMERGE) to probe the dependencies of pair merging probabilities and merging timescales. Also show that some commonly used pair selection criteria may not represent a suitable sample of galaxies to reproduce underlying merger rates. Seem to talk about both observation timescales and merger timescales, and I'm not sure which one they focus on.
    Quantified probability of a "close pair" merging before z=0 and dependence on rsep-proj, LOS vel, and z. 
    On what timescale will an observed pair merge, and for how long is that pair observable given the selection criteria. 
    What determines pair observation timescales (might be a followup from their 2020 paper) 
    Notes that papers that attempt to come up with fitting functions for timescales etc. will all be wrong for pairs w different selection criteria.
    -> Maybe motivates that we should say that future extensions of our work may lead to the development of fitting functions that can be applied to any system? (is this true?)
    

    
    Author: Ventou2017
    Paper: 
    What's the connection: 
    Other notes: 
    
    Author: Patton2024
    Paper: 
    What's the connection:
    Other notes:
    

    Author: Guzman-Ortega 2022
    Paper: Morphological signatures of mergers in the TNG50 simulation and the Kilo-Degree Survey: the merger fraction from dwarfs to Milky Way-like galaxies
    What's the connection:
    Other notes:


OTHER:
    Author: Wang2020 (+Pearson, Rodriguez-Gomez)
    Paper: Towards a consistent framework of comparing galaxy mergers in observations and simulations
    What's the connection: Looked at the major merger fraction of stuff at low-z. Their stellar mass criteria did not cover our range of interest for low-mass systems
    Other notes: Mass dependence of the galaxy major merger fraction at low z. Make synthetic training images from TNG and apply to KiDS images from GAMA. Find that major merger fraction decreases slightly with increasing stellar mass (1e9.5-1e11.5) -> 
    major mergers defined as happening in the next 1Gyr or in the previous 500Myr. 
    
    Paper: 
    Author: RG2017
    What's the connection: 
    Other notes: 
    
    
    
    Paper: 
    Author: Patton2020
    What's the connection:
    Other notes:
    
    Paper: 
    Author: Patton2000
    What's the connection:
    Other notes:
    
    
    
    Paper: Emerge: Empirical predictions of galaxy merger rates since $z \sim 6$ 
    Author: O'Leary 2020
    What's the connection:
    Other notes:
    
    
    
    
    Paper: SATELLITES IN MILKY-WAY-LIKE HOSTS: ENVIRONMENT DEPENDENCE AND CLOSE PAIRS
    Author: González2013
    What's the connection:
    Other notes:
    
    Paper: Observational Constraints on the Merger History of Galaxies since $z\sim6$: Probabilistic Galaxy Pair Counts in the CANDELS Fields
    Author: Duncan 2019
    What's the connection:
    Other notes:
    
    Paper: THE MILLENNIUM GALAXY CATALOGUE: THE CONNECTION BETWEEN CLOSE PAIRS AND ASYMMETRY; IMPLICATIONS FOR THE GALAXY MERGER RATE
    Author: De Propris 2007
    What's the connection:
    Other notes:
    
    Paper: Galaxy interactions in IllustrisTNG-100, I: The power and limitations of visual identification
    Author: Blumenthal 2020
    What's the connection:
    Other notes:
    
    Paper: Small-Scale Challenges to the ΛCDM Paradigm
    Author: Bullock 2017
    What's the connection:
    Other notes:
    
    Paper:
    Author:
    What's the connection:
    Other notes:
    
    Paper:
    Author:
    What's the connection:
    Other notes:
    
    Paper:
    Author:
    What's the connection:
    Other notes:
    
    Paper:
    Author:
    What's the connection:
    Other notes: